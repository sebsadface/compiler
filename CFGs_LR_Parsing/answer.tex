%%%%%%%%%%%%%%%%%%%%% PACKAGE IMPORTS %%%%%%%%%%%%%%%%%%%%%
\documentclass[11pt]{article}
\usepackage{amsmath, amsfonts, amsthm, amssymb}
\usepackage{lmodern}
\usepackage{microtype}
\usepackage{fullpage}       
\usepackage{changepage}
\usepackage{hyperref}
\usepackage{blindtext}
\usepackage{mathtools}
\hypersetup{
    colorlinks=true,
    linkcolor=blue,
    filecolor=magenta,      
    urlcolor=blue,
    pdftitle={Overleaf Example},
    pdfpagemode=FullScreen,
    }
\urlstyle{same}

\newenvironment{level}%
{\addtolength{\itemindent}{2em}}%
{\addtolength{\itemindent}{-2em}}

\usepackage{amsmath,amsthm,amssymb}


\usepackage[x11names, rgb]{xcolor}
\usepackage{graphicx}
\usepackage[nooldvoltagedirection]{circuitikz}
\usetikzlibrary{decorations,arrows,shapes}

\usepackage{datetime}
\usepackage{etoolbox}
\usepackage{enumerate}
\usepackage{enumitem}
\usepackage{listings}
\usepackage{array}
\usepackage{varwidth}
\usepackage{tcolorbox}
\usepackage{amsmath}
\usepackage{circuitikz}
\usepackage{verbatim}
\usepackage[linguistics]{forest}
\usepackage{listings}
\usepackage{xcolor}
\renewcommand{\rmdefault}{cmss}


\newcommand\doubleplus{+\kern-1.3ex+\kern0.8ex}
\newcommand\mdoubleplus{\ensuremath{\mathbin{+\mkern-10mu+}}}

\definecolor{codegreen}{rgb}{0,0.6,0}
\definecolor{codegray}{rgb}{0.5,0.5,0.5}
\definecolor{codepurple}{rgb}{0.58,0,0.82}
\definecolor{backcolour}{rgb}{0.95,0.95,0.92}

\lstdefinelanguage{JavaScript}{
  keywords={typeof, new, true, false, catch, function, return, null, catch, switch, var, if, in, while, do, else, case, break},
  keywordstyle=\color{blue}\bfseries,
  ndkeywords={class, export, boolean, throw, implements, import, this},
  ndkeywordstyle=\color{darkgray}\bfseries,
  identifierstyle=\color{black},
  sensitive=false,
  comment=[l]{//},
  morecomment=[s]{/*}{*/},
  commentstyle=\color{purple}\ttfamily,
  stringstyle=\color{red}\ttfamily,
  morestring=[b]',
  morestring=[b]"
}

\lstdefinestyle{mystyle}{
    language=JavaScript,
    backgroundcolor=\color{backcolour},   
    commentstyle=\color{codegreen},
    keywordstyle=\color{magenta},
    numberstyle=\tiny\color{codegray},
    stringstyle=\color{codepurple},
    basicstyle=\ttfamily\footnotesize,
    breakatwhitespace=false,         
    breaklines=true,                 
    captionpos=b,                    
    keepspaces=true,                 
    numbers=left,                    
    numbersep=5pt,                  
    showspaces=false,                
    showstringspaces=false,
    showtabs=false,                  
    tabsize=2
}

\lstset{style=mystyle}
\setlength{\parindent}{0pt}
\setlength{\parskip}{5pt plus 1pt}

\providetoggle{questionnumbers}
\settoggle{questionnumbers}{true}
\newcommand{\noquestionnumbers}{
    \settoggle{questionnumbers}{false}
}

\newcounter{questionCounter}
\newenvironment{question}[2][\arabic{questionCounter}]{%
    \ifnum\value{questionCounter}=0 \else {\newpage}\fi%
    \setcounter{partCounter}{0}%
    \vspace{.25in} \hrule \vspace{0.5em}%
    \noindent{\bf \iftoggle{questionnumbers}{Question #1: }{}#2}%
    \addtocounter{questionCounter}{1}%
    \vspace{0.8em} \hrule \vspace{.10in}%
}

\newcounter{partCounter}[questionCounter]
\renewenvironment{part}[1][\alph{partCounter}]{%
    \addtocounter{partCounter}{1}%
    \vspace{.10in}%
    \begin{indented}%
       {\bf (#1)} %
}{\end{indented}}

\def\indented#1{\list{}{}\item[]}
\let\indented=\endlist
\def\show#1{\ifdefempty{#1}{}{#1\\}}
\def\IMP{\rightarrow}
\def\AND{\wedge}
\def\OR{\vee}
\def\BI{\leftrightarrow}
\def\DIFF{\setminus}
\def\SUB{\subseteq}


\newcolumntype{C}{>{\centering\arraybackslash}m{1.5cm}}
\renewcommand\qedsymbol{$\blacksquare$}
\newtcolorbox{answer}
{
  colback   = green!5!white,    % Background color
  colframe  = green!75!black,   % Outline color
  box align = center,           % Align box on text line
  varwidth upper,               % Enables multi line input
  hbox                          % Bounds box to text width
}

\newcommand{\myhwname}{Homework 1}
\newcommand{\myname}{Sebastian Liu}
\newcommand{\myemail}{ll57@cs.washington.edu}
\newcommand{\mysection}{AB}
\newcommand{\dollararrow}{\stackrel{\$}{\leftarrow}}
%%%%%%%%%%%%%%%%%%%%%%%%%%%%%%%%%%%%%%%%%%%%%%%%%%%%%%%%%%%

%%%%%%%%%%%%%%%%%%% Document Options %%%%%%%%%%%%%%%%%%%%%%
\noquestionnumbers
%%%%%%%%%%%%%%%%%%%%%%%%%%%%%%%%%%%%%%%%%%%%%%%%%%%%%%%%%%%

%%%%%%%%%%%%%%%%%%%%%%%% WORK BELOW %%%%%%%%%%%%%%%%%%%%%%%%
\begin{document}

\begin{center}
    \textbf{Homework 2 - CFGs and LR Parsing} \bigskip
\end{center}

%%%%%%%%%%%%%%%%%%%%%%%% Task 1 %%%%%%%%%%%%%%%%%%%%%%%%M
\begin{question}{Question 1} 
    \begin{part}
        \begin{answer}
            $$ \textbf{Derivation 1:  }S \rightarrow aSbS 
                \rightarrow abS
                \rightarrow abaSbS
                \rightarrow ababS
                \rightarrow abab$$

            $$ \textbf{Derivation 2:  }S \rightarrow aSbS 
            \rightarrow abSaSbS
            \rightarrow abaSbS
            \rightarrow ababS
            \rightarrow abab
            $$
        \end{answer}
    \end{part}

    \begin{part}
        \begin{answer}
            $$ \textbf{Derivation 1:  }S \rightarrow aSbS 
            \rightarrow aSbaSbS
            \rightarrow aSbaSb
            \rightarrow aSbab
            \rightarrow abab
            $$

            $$ \textbf{Derivation 2:  }S \rightarrow aSbS 
            \rightarrow aSb
            \rightarrow abSaSb
            \rightarrow abSab
            \rightarrow abab
            $$
        \end{answer}
    \end{part}

    \begin{part}
        \begin{answer}
            \textbf{Note:} parse tree for derivation 1 in both leftmost and rightmost are the same (shown on the left), and  the parse tree for derivation 2 in both leftmost and rightmost are the same (shown on the right). \\
            \textbf{Parse tree for Derivation 1s}: \;\;\;\;\;\;\;\;\;\;\;\;\;\;\;\;\;\;\;\;\;\;\;\;\;\;\;\;\;\;\;\; \textbf{Parse tree for Derivation 2s}: \\
            \includegraphics*[width=0.56\linewidth]{1deriv1.png}
            \includegraphics*[width=0.46\linewidth]{1deriv2.png}
        \end{answer}
    \end{part}
\end{question}

%%%%%%%%%%%%%%%%%%%%%%%% Task 2 %%%%%%%%%%%%%%%%%%%%%%%%
\begin{question}{Question 2}
    \begin{part}
        \begin{answer}
            \textbf{Left-most derivation:}
            \begin{align*}
                S \rightarrow (L) 
            \rightarrow (L,S)
            \rightarrow (S, S)
            \rightarrow (x, S)
            \rightarrow (x, (L))
            \rightarrow (x, (L,S))
            \rightarrow (x, (S, S))
            &\rightarrow (x, (x, S))\\
            &\rightarrow (x, (x, x))
            \end{align*}
        \end{answer}
    \end{part}

    \begin{part}
        \begin{answer}
            \textbf{Right-most derivation:}
            \begin{align*}
                S \rightarrow (L) 
            \rightarrow (L,S)
            \rightarrow (L,(L))
            \rightarrow (L,(L,S))
            \rightarrow (L,(L,x))
            \rightarrow (L,(S,x))
            &\rightarrow (L,(x,x))\\
            &\rightarrow (S,(x,x))\\
            &\rightarrow (x,(x,x))
            \end{align*}
        \end{answer}
    \end{part}

    \begin{part}
        \begin{answer}
            \begin{tabular}[h]{lll}
                \textbf{Stack} & \textbf{Input} & \textbf{Action} \\
                \hline
                \$     & $(x,x,x)$ \$ & shift  \\
                \$$($    &  $x,x,x)$ \$ & shift  \\
                \$$(x$   &  $,x,x)$  \$ & reduce \\
                \$$(S$   &   $,x,x)$ \$ & reduce \\
                \$$(L$   &  $,x,x)$  \$ & shift  \\
                \$$(L,$  &   $x,x)$  \$ & shift  \\
                \$$(L,x$ &   $,x)$   \$ & reduce \\
                \$$(L,S$ &   $,x)$   \$ & reduce \\
                \$$(L$  &    $,x) $  \$ & shift  \\
                \$$(L,$  &   $ x) $  \$ & shift  \\
                \$$(L,x$ &   $ )  $  \$ & reduce \\
                \$$(L,S$ &   $ ) $   \$ & reduce \\
                \$$(L$   &    $)$    \$ & shift  \\
                \$$(L)$  &         \$ & reduce \\
                \$$S$ &            \$ & accept \\
            \end{tabular}
        \end{answer}
    \end{part}
\newpage
    \begin{part}
        \begin{answer}
            \begin{tabular}[h]{lll}
                \textbf{Stack} & \textbf{Input} & \textbf{Action} \\
                \hline
                \$       & $(x,x,x)$ \$ & shift  \\
                \$$(    $  &  $x,x,x)$ \$ & shift  \\
                \$$(x     $&  $,x,x)$  \$ & reduce \\
                \$$(S     $&   $,x,x)$ \$ & shift \\
                \$$(S,    $&    $x,x)$ \$ & shift \\
                \$$(S,x  $ &     $,x)$ \$ & reduce \\
                \$$(S,S  $ &     $,x)$ \$ & shift \\
                \$$(S,S, $ &     $ x)$ \$ & shift \\
                \$$(S,S,x $&      $ )$ \$ & reduce \\
                \$$(S,S,S $&       $)$ \$ & reduce \\
                \$$(S,S,L$ &       $)$ \$ & reduce \\
                \$$(S,L$   &       $)$ \$ & reduce \\
                \$$(L$     &       $)$ \$ & shift \\
                \$$(L)$    &        \$ & reduce \\
                \$ $S $    &        \$ & accept \\
            \end{tabular}\\\\
            The depth of the stack during self-reduce parse \textbf{increases} if we replace the left-recursive production $L\Coloneq L,S$ (max depth: 4) with right-recursive production $L \Coloneq S,L$ (max depth: 6).
        \end{answer}
    \end{part}
\end{question}

%%%%%%%%%%%%%%%%%%%%%%%% Task 3 %%%%%%%%%%%%%%%%%%%%%%%%
\begin{question}{Question 3}
    \begin{part}
        \begin{answer}
            \textbf{Non-terminals:} $S$ (sentence), $NP$ (noun phrase), $VP$ (verb phrase), $PP$ (prepositional phrase), $N$ (noun), $V$ (verb), $Art$ (article), $P$ (preposition) \\\\
            \textbf{Grammar Rules:} \\
            $S \Coloneq S ; S \mid NP\;\; VP$ \\
            $NP \Coloneq Art \;\; N \mid N \;\; N \mid N$ \\
            $VP \Coloneq V \;\; NP \mid V \;\; PP$ \\
            $PP \Coloneq P \;\; NP$ \\
            $N \Coloneq \mathrm{time} \mid \mathrm{arrow} \mid \mathrm{banana} \mid \mathrm{fruit} \mid \mathrm{flies} $\\
            $V \Coloneq \mathrm{flies} \mid \mathrm{like}$ \\
            $Art \Coloneq \mathrm{a} \mid \mathrm{an} \mid \mathrm{the}$ \\
            $P \Coloneq \mathrm{like}$ \\\\
            \textbf{Parse tree \#1}: \;\;\;\;\;\;\;\;\;\;\;\;\;\;\;\;\;\;\;\;\;\;\;\;\;\;\;\;\;\;\;\;\;\;\;\;\;\;\;\;\;\;\;\;\;\;\; \textbf{Parse tree \# 2}: \\
            \includegraphics*[width=0.51\linewidth]{3.png}
            \includegraphics*[width=0.51\linewidth]{3_2.png}
        \end{answer}
    \end{part}
\end{question}

%%%%%%%%%%%%%%%%%%%%%%%% Task 4 %%%%%%%%%%%%%%%%%%%%%%%%
\begin{question}{Question 4}
    \begin{part}
        \begin{answer}
            We add a production $E’$ with the epithets symbol $E$ followed by end of file (\$), so we have grammar rule:\\
            $\textbf{0.  } E’ \Coloneq E\;\$ $ \\
            $\textbf{1.  } E \Coloneq D $\;\;\;\;\;\;\;\;\;\;\;\;\;\;\;$\textbf{5.  } T \Coloneq \textrm{quick} $\\
            $\textbf{2.  } D \Coloneq T $\;\;\;\;\;\;\;\;\;\;\;\;\;\;\;$\textbf{6.  } T \Coloneq \textrm{strong}$\\
            $\textbf{3.  } D \Coloneq S $\;\;\;\;\;\;\;\;\;\;\;\;\;\;\;$\textbf{7.  } A \Coloneq \textrm {bunny}$\\
            $\textbf{4.  } S \Coloneq T \textrm{ like } A $\;\;\;\;\;\;$\textbf{8.  } A \Coloneq \textrm{ox}$\\\\
            We have the following LR(0) state diagram:\\
            \includegraphics*[width=0.8\linewidth]{4a.png} \\
            and parse table:\\
            \includegraphics*[width=\linewidth]{4ab.png} 
        \end{answer}
    \end{part}
\newpage
    \begin{part}
        \begin{answer}
            \textbf{FIRST, FOLLOW, nullable for each non-terminal:} \\
            \includegraphics*[width=\linewidth]{4b.png} 
        \end{answer}
    \end{part}

    \begin{part}
        \begin{answer}
            \textbf{SLR parse table:} \\
            \includegraphics*[width=\linewidth]{4c.png} 
        \end{answer}
    \end{part}

    \begin{part}
        \begin{answer}
            This grammar is not LR(0) because it has shift-reduce conflicts in state 3 without lookaheads (as shown in the LR(0) parse table).
            This shift-reduce conflict in state 3 is resolved in the SLR parse table by using the FOLLOW sets to determine when a reduction should occur (as shown in the SLR parse table).
            Therefore, this grammar is \textbf{SLR}.
        \end{answer}
    \end{part}
\end{question}
\end{document}