%%%%%%%%%%%%%%%%%%%%% PACKAGE IMPORTS %%%%%%%%%%%%%%%%%%%%%
\documentclass[11pt]{article}
\usepackage{amsmath, amsfonts, amsthm, amssymb}
\usepackage{lmodern}
\usepackage{microtype}
\usepackage{fullpage}       
\usepackage{changepage}
\usepackage{hyperref}
\usepackage{blindtext}
\usepackage{mathtools}
\hypersetup{
    colorlinks=true,
    linkcolor=blue,
    filecolor=magenta,      
    urlcolor=blue,
    pdftitle={Overleaf Example},
    pdfpagemode=FullScreen,
    }
\urlstyle{same}

\newenvironment{level}%
{\addtolength{\itemindent}{2em}}%
{\addtolength{\itemindent}{-2em}}

\usepackage{amsmath,amsthm,amssymb}


\usepackage[x11names, rgb]{xcolor}
\usepackage{graphicx}
\usepackage[nooldvoltagedirection]{circuitikz}
\usetikzlibrary{decorations,arrows,shapes}

\usepackage{datetime}
\usepackage{etoolbox}
\usepackage{enumerate}
\usepackage{enumitem}
\usepackage{listings}
\usepackage{array}
\usepackage{varwidth}
\usepackage{tcolorbox}
\usepackage{amsmath}
\usepackage{circuitikz}
\usepackage{verbatim}
\usepackage[linguistics]{forest}
\usepackage{listings}
\usepackage{xcolor}
\renewcommand{\rmdefault}{cmss}


\newcommand\doubleplus{+\kern-1.3ex+\kern0.8ex}
\newcommand\mdoubleplus{\ensuremath{\mathbin{+\mkern-10mu+}}}

\definecolor{codegreen}{rgb}{0,0.6,0}
\definecolor{codegray}{rgb}{0.5,0.5,0.5}
\definecolor{codepurple}{rgb}{0.58,0,0.82}
\definecolor{backcolour}{rgb}{0.95,0.95,0.92}

\lstdefinelanguage{JavaScript}{
  keywords={typeof, new, true, false, catch, function, return, null, catch, switch, var, if, in, while, do, else, case, break},
  keywordstyle=\color{blue}\bfseries,
  ndkeywords={class, export, boolean, throw, implements, import, this},
  ndkeywordstyle=\color{darkgray}\bfseries,
  identifierstyle=\color{black},
  sensitive=false,
  comment=[l]{//},
  morecomment=[s]{/*}{*/},
  commentstyle=\color{purple}\ttfamily,
  stringstyle=\color{red}\ttfamily,
  morestring=[b]',
  morestring=[b]"
}

\lstdefinestyle{mystyle}{
    language=JavaScript,
    backgroundcolor=\color{backcolour},   
    commentstyle=\color{codegreen},
    keywordstyle=\color{magenta},
    numberstyle=\tiny\color{codegray},
    stringstyle=\color{codepurple},
    basicstyle=\ttfamily\footnotesize,
    breakatwhitespace=false,         
    breaklines=true,                 
    captionpos=b,                    
    keepspaces=true,                 
    numbers=left,                    
    numbersep=5pt,                  
    showspaces=false,                
    showstringspaces=false,
    showtabs=false,                  
    tabsize=2
}

\lstset{style=mystyle}
\setlength{\parindent}{0pt}
\setlength{\parskip}{5pt plus 1pt}

\providetoggle{questionnumbers}
\settoggle{questionnumbers}{true}
\newcommand{\noquestionnumbers}{
    \settoggle{questionnumbers}{false}
}

\newcounter{questionCounter}
\newenvironment{question}[2][\arabic{questionCounter}]{%
    \ifnum\value{questionCounter}=0 \else {\newpage}\fi%
    \setcounter{partCounter}{0}%
    \vspace{.25in} \hrule \vspace{0.5em}%
    \noindent{\bf \iftoggle{questionnumbers}{Question #1: }{}#2}%
    \addtocounter{questionCounter}{1}%
    \vspace{0.8em} \hrule \vspace{.10in}%
}

\newcounter{partCounter}[questionCounter]
\renewenvironment{part}[1][\alph{partCounter}]{%
    \addtocounter{partCounter}{1}%
    \vspace{.10in}%
    \begin{indented}%
       {\bf (#1)} %
}{\end{indented}}

\def\indented#1{\list{}{}\item[]}
\let\indented=\endlist
\def\show#1{\ifdefempty{#1}{}{#1\\}}
\def\IMP{\rightarrow}
\def\AND{\wedge}
\def\OR{\vee}
\def\BI{\leftrightarrow}
\def\DIFF{\setminus}
\def\SUB{\subseteq}


\newcolumntype{C}{>{\centering\arraybackslash}m{1.5cm}}
\renewcommand\qedsymbol{$\blacksquare$}
\newtcolorbox{answer}
{
  colback   = green!5!white,    % Background color
  colframe  = green!75!black,   % Outline color
  box align = center,           % Align box on text line
  varwidth upper,               % Enables multi line input
  hbox                          % Bounds box to text width
}

\newcommand{\myhwname}{Homework 3}
\newcommand{\myname}{Sebastian Liu}
\newcommand{\myemail}{ll57@cs.washington.edu}
\newcommand{\mysection}{AB}
\newcommand{\dollararrow}{\stackrel{\$}{\leftarrow}}
%%%%%%%%%%%%%%%%%%%%%%%%%%%%%%%%%%%%%%%%%%%%%%%%%%%%%%%%%%%

%%%%%%%%%%%%%%%%%%% Document Options %%%%%%%%%%%%%%%%%%%%%%
\noquestionnumbers
%%%%%%%%%%%%%%%%%%%%%%%%%%%%%%%%%%%%%%%%%%%%%%%%%%%%%%%%%%%

%%%%%%%%%%%%%%%%%%%%%%%% WORK BELOW %%%%%%%%%%%%%%%%%%%%%%%%
\begin{document}

\begin{center}
    \textbf{Homework 3 - LL Grammars} \bigskip
\end{center}

%%%%%%%%%%%%%%%%%%%%%%%% Task 1 %%%%%%%%%%%%%%%%%%%%%%%%M
\begin{question}{Question 1} 
        \begin{answer}
            The grammar does not satisfy the LL(1) condition, because it contains an indirect left recursion, where the grammar recursively alternates between
            $B$ and $C$.\\\\
            \textbf{Solution:}\\
            Substitue B into C:\\
            $1.\;\; \textrm{C} \Coloneq \textrm{C }  \texttt{ri} \mid \texttt{ti} \mid \texttt{t}$\\
            To resolve the left recursion, we create recursive tail from suffix of recursive production:\\
            $2. \;\; \textrm{RTail} \Coloneq \texttt{ri} \textrm{ RTail}$\\
            Append Tail to non-recursive productions:\\
            $1. \;\;\textrm{C} \Coloneq \texttt{ti} \textrm{ RTail} \mid \texttt{t} \textrm{ RTail}$\\
            $2. \;\; \textrm{RTail} \Coloneq \texttt{ri} \textrm{ RTail}$\\
            Add empty string ($\epsilon$) as a rhs for the tail production:\\
            $1. \;\;\textrm{C} \Coloneq \texttt{ti} \textrm{ RTail} \mid \texttt{t} \textrm{ RTail}$\\
            $2. \;\; \textrm{RTail} \Coloneq \texttt{ri} \textrm{ RTail} \mid \epsilon$\\
            To resolve the FIRST conflict in C, we factor out the prefix \texttt{t}:\\
            $1. \;\;\textrm{C} \Coloneq \texttt{t} \textrm{ FTail} $\\
            $2. \;\; \textrm{RTail} \Coloneq \texttt{ri} \textrm{ RTail} \mid \epsilon$\\
            $3.\;\; \textrm{FTail} \Coloneq \texttt{i} \textrm{ RTail} \mid \textrm{RTail}$\\
            Reorder the productions to make the grammar more readable:\\
            $1. \;\;\textrm{C} \Coloneq \texttt{t} \textrm{ FTail} $\\
            $2.\;\; \textrm{FTail} \Coloneq \texttt{i} \textrm{ RTail} \mid \textrm{RTail}$\\
            $3. \;\; \textrm{RTail} \Coloneq \texttt{ri} \textrm{ RTail} \mid \epsilon$\\

            \textbf{The new grammar is:}\\
            $0. \;\; \textrm{A} \Coloneq \texttt{s } \textrm{C } \texttt{ng} \mid \epsilon$ \\
            $1. \;\;\textrm{C} \Coloneq \texttt{t} \textrm{ FTail} $\\
            $2.\;\; \textrm{FTail} \Coloneq \texttt{i} \textrm{ RTail} \mid \textrm{RTail}$\\
            $3. \;\; \textrm{RTail} \Coloneq \texttt{ri} \textrm{ RTail} \mid \epsilon$\\

        \end{answer}
\end{question}

%%%%%%%%%%%%%%%%%%%%%%%% Task 2 %%%%%%%%%%%%%%%%%%%%%%%%
\begin{question}{Question 2}
        \begin{answer}
            $0. \;\;\textrm{S} \Coloneq \textrm{S} \texttt{;} \textrm{S} \mid \texttt{id} \texttt{:=} \textrm{ E} \mid \texttt{print} \texttt{(} \textrm{L} \texttt{)} $\\
            $1. \;\; \textrm{E} \Coloneq \texttt{id} \mid \texttt{num} \mid \textrm {E \texttt{+} E} \mid \texttt{(}\textrm{S} \texttt{,} \textrm{E} \texttt{)}$\\
            $2. \;\;\textrm{L} \Coloneq \textrm{E} \mid \textrm{L} \texttt{,} \textrm{E}$\\\\
            Resolve the left recursion in S, E, and L:\\
            Create recursive tail from suffix of recursive production:\\
            $3. \;\; \textrm{STail} \Coloneq \texttt{;} \textrm{S} \textrm{ STail}$\\
            $4. \;\; \textrm{ETail} \Coloneq \texttt{+} \textrm{E} \textrm{ ETail}$\\
            $5. \;\; \textrm{LTail} \Coloneq \texttt{,} \textrm{E} \textrm{ LTail}$\\
            Append Tail to non-recursive productions and add empty string ($\epsilon$) as a rhs for the tail production:\\
            $0. \;\;\textrm{S} \Coloneq \texttt{id} \texttt{:=} \textrm{ E} \textrm{ STail} \mid \texttt{print} \texttt{(} \textrm{L} \texttt{)} \textrm{ STail}$\\
            $1. \;\; \textrm{E} \Coloneq \texttt{id} \textrm{ ETail} \mid \texttt{num} \textrm{ ETail} \mid \texttt{(}\textrm{S} \texttt{,} \textrm{E} \texttt{)} \textrm{ ETail}$\\
            $2. \;\;\textrm{L} \Coloneq \textrm{E} \textrm{ LTail}$\\
            $3. \;\; \textrm{STail} \Coloneq \texttt{;} \textrm{S} \textrm{ STail} \mid \epsilon$\\
            $4. \;\; \textrm{ETail} \Coloneq \texttt{+} \textrm{E} \textrm{ ETail} \mid \epsilon$\\
            $5. \;\; \textrm{LTail} \Coloneq \texttt{,} \textrm{E} \textrm{ LTail} \mid \epsilon$\\

            \textbf{The new grammar is:}\\
            $0. \;\;\textrm{S} \Coloneq \texttt{id} \texttt{:=} \textrm{ E} \textrm{ STail} \mid \texttt{print} \texttt{(} \textrm{L} \texttt{)} \textrm{ STail}$\\
            $1. \;\; \textrm{E} \Coloneq \texttt{id} \textrm{ ETail} \mid \texttt{num} \textrm{ ETail} \mid \texttt{(}\textrm{S} \texttt{,} \textrm{E} \texttt{)} \textrm{ ETail}$\\
            $2. \;\;\textrm{L} \Coloneq \textrm{E} \textrm{ LTail}$\\
            $3. \;\; \textrm{STail} \Coloneq \texttt{;} \textrm{S} \textrm{ STail} \mid \epsilon$\\
            $4. \;\; \textrm{ETail} \Coloneq \texttt{+} \textrm{E} \textrm{ ETail} \mid \epsilon$\\
            $5. \;\; \textrm{LTail} \Coloneq \texttt{,} \textrm{E} \textrm{ LTail} \mid \epsilon$\\
        \end{answer}
\end{question}
\end{document}